\documentclass[10pt,a4paper]{article}
\usepackage[utf8]{inputenc}
\usepackage{amsmath}
\usepackage{amsfonts}
\usepackage{graphicx}
\usepackage{amssymb}
\usepackage{fullpage}
\title{Cleaning, Disinfection, and Mapping Robot}
\author{Brian Lauer\and Suruz Miah}

\begin{document}
\maketitle

Nowadays, many different robots are available on the market for the purpose of
ultraviolet (UV) disinfection from companies like UVD
Robots\footnote{http://www.uvd-robots.com/}. These robots typically employ
sensors, such as ultrasonic sensors for ranging, LIDAR sensors or cameras for
creating a map of a room, and cliff sensors for recognition of
drops or edges. One of the main features of these robots is to utilize UVC lamp that is
capable of performing efficient elimination of many harmful microorganisms
including bacteria and viruses within a room through DNA destruction and
disruption. However, a limitation to these current robots on the market is the
inability to shine UV light vertically: only horizontally with full 360$^\circ$
range of emission. The proposed robot design shown in Figure~\ref{camRobot} will
support emission of light horizontally as well as vertically by including two
horizontally positioned lamps able to disinfect areas directly above the
robot. %
%
\begin{figure}[htbp]
  \centering
  \includegraphics[scale=0.5]{figs/img/frontViewScreenshotB}
  \caption{Current cleaning, disinfection, and mapping robot design}
  \label{camRobot}
\end{figure}
%
Secondly, because UV light is deemed dangerous to humans and possibly other
animals a disinfection sprayer will help to further reduce the concentration of
harmful pathogens. In this way, full room disinfection can be accounted for in
case any live animals are unable to be relocated during the disinfection period.
During disinfection of a room with UVC light such as a hospital room, no humans
or animals may be present. To provide a maximum turning radius and stability,
the robot features a two-wheel differential drive design with two additional
points of contact via two caster wheels. To generate a digital map, a LIDAR
sensor is present in the center of the robot which will relay serial data to the
robot's central computer, a Beaglebone single board computer with the SLAM
(Simultaneous Localization and Mapping) algorithm.

Another feature that the proposed robot will boast is the presence of a
vacuuming mechanism capable of removing dust particles or debris from floor
surfaces. In this way, a room can be thoroughly cleaned with little to no human
intervention which is in favor due to the current health crisis. This robot will
be able to effectively combine the features of the robot designs by iRobot and
UVD robots. Another feature that this robot will propose to implement is the
integration of a mobile application capable of wireless communication with the
onboard computer of the robot to direct the robot to perform various actions
like directing the robot to a specific room or area of a home or building.
Secondly, a real time 2D map of the room will be visible to allow owners to view
the current progress of the cleaning and disinfection process. Lastly, 14
ultrasonic sensors will be present around the circumference of the robot chassis
to perform additional ranging and obstacle avoidance.
\end{document}

%%% Local Variables:
%%% mode: latex
%%% TeX-master: t
%%% End:
