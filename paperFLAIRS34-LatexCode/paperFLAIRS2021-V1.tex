\def\year{2021}\relax
%File: formatting-instruction.tex
\documentclass[letterpaper]{article} % DO NOT CHANGE THIS
\usepackage{aaai20}  % DO NOT CHANGE THIS
\usepackage{times}  % DO NOT CHANGE THIS
\usepackage{helvet} % DO NOT CHANGE THIS
\usepackage{courier}  % DO NOT CHANGE THIS
\usepackage[hyphens]{url}  % DO NOT CHANGE THIS
\usepackage{graphicx} % DO NOT CHANGE THIS
\urlstyle{rm} % DO NOT CHANGE THIS
\def\UrlFont{\rm}  % DO NOT CHANGE THIS

\usepackage{amssymb,amsmath,bm}
\usepackage{graphics,adjustbox}
\usepackage{tikz}
\usepackage{subcaption}
\usepackage{siunitx}
\sisetup{unitsep = \cdot}

\usepackage[figuresright]{rotating}
\usepackage[colorinlistoftodos]{todonotes}
\usepackage[english,algo2e,algoruled,vlined,linesnumbered]{algorithm2e}   % package for algorithm
\usepackage{enumerate}

\usepackage{easyReview}
\newtheorem{assumption}{Assumption}


\DeclareMathOperator*{\argmin}{arg min}


\frenchspacing  % DO NOT CHANGE THIS
\setlength{\pdfpagewidth}{8.5in}  % DO NOT CHANGE THIS
\setlength{\pdfpageheight}{11in}  % DO NOT CHANGE THIS
%\nocopyright
%PDF Info Is REQUIRED.
% For /Author, add all authors within the parentheses, separated by commas. No accents or commands.
% For /Title, add Title in Mixed Case. No accents or commands. Retain the parentheses.
 \pdfinfo{
/Title (Smart Sanitizing Indoor Mobile Robot: Design and Development)
/Author (Brian Lauer, Nicoulas Shepard, Fazel Keshtkar, and Md Suruz Miah)
} %Leave this	
% /Title ()
% Put your actual complete title (no codes, scripts, shortcuts, or LaTeX commands) within the parentheses in mixed case
% Leave the space between \Title and the beginning parenthesis alone
% /Author ()
% Put your actual complete list of authors (no codes, scripts, shortcuts, or LaTeX commands) within the parentheses in mixed case. 
% Each author should be only by a comma. If the name contains accents, remove them. If there are any LaTeX commands, 
% remove them. 

% DISALLOWED PACKAGES
% \usepackage{authblk} -- This package is specifically forbidden
% \usepackage{balance} -- This package is specifically forbidden
% \usepackage{caption} -- This package is specifically forbidden
% \usepackage{color (if used in text)
% \usepackage{CJK} -- This package is specifically forbidden
% \usepackage{float} -- This package is specifically forbidden
% \usepackage{flushend} -- This package is specifically forbidden
% \usepackage{fontenc} -- This package is specifically forbidden
% \usepackage{fullpage} -- This package is specifically forbidden
% \usepackage{geometry} -- This package is specifically forbidden
% \usepackage{grffile} -- This package is specifically forbidden
% \usepackage{hyperref} -- This package is specifically forbidden
% \usepackage{navigator} -- This package is specifically forbidden
% (or any other package that embeds links such as navigator or hyperref)
% \indentfirst} -- This package is specifically forbidden
% \layout} -- This package is specifically forbidden
% \multicol} -- This package is specifically forbidden
% \nameref} -- This package is specifically forbidden
% \natbib} -- This package is specifically forbidden -- use the following workaround:
% \usepackage{savetrees} -- This package is specifically forbidden
% \usepackage{setspace} -- This package is specifically forbidden
% \usepackage{stfloats} -- This package is specifically forbidden
% \usepackage{tabu} -- This package is specifically forbidden
% \usepackage{titlesec} -- This package is specifically forbidden
% \usepackage{tocbibind} -- This package is specifically forbidden
% \usepackage{ulem} -- This package is specifically forbidden
% \usepackage{wrapfig} -- This package is specifically forbidden
% DISALLOWED COMMANDS
% \nocopyright -- Your paper will not be published if you use this command
% \addtolength -- This command may not be used
% \balance -- This command may not be used
% \baselinestretch -- Your paper will not be published if you use this command
% \clearpage -- No page breaks of any kind may be used for the final version of your paper
% \columnsep -- This command may not be used
% \newpage -- No page breaks of any kind may be used for the final version of your paper
% \pagebreak -- No page breaks of any kind may be used for the final version of your paperr
% \pagestyle -- This command may not be used
% \tiny -- This is not an acceptable font size.
% \vspace{- -- No negative value may be used in proximity of a caption, figure, table, section, subsection, subsubsection, or reference
% \vskip{- -- No negative value may be used to alter spacing above or below a caption, figure, table, section, subsection, subsubsection, or reference

\setcounter{secnumdepth}{0} %May be changed to 1 or 2 if section numbers are desired.

% The file aaai20.sty is the style file for AAAI Press 
% proceedings, working notes, and technical reports.
%
\setlength\titlebox{2.5in} % If your paper contains an overfull \vbox too high warning at the beginning of the document, use this
% command to correct it. You may not alter the value below 2.5 in
\title{Smart Sanitizing Mobile Robot using Ultra-violet Sterilization
  Technology: A Prototype}
%Your title must be in mixed case, not sentence case. 
% That means all verbs (including short verbs like be, is, using,and go), 
% nouns, adverbs, adjectives should be capitalized, including both words in hyphenated terms, while
% articles, conjunctions, and prepositions are lower case unless they
% directly follow a colon or long dash
\author{~%
Brian Lauer\textsuperscript{\rm 1}, Nicoulas Shepard\textsuperscript{\rm 1},
Fazel Keshtkar\textsuperscript{\rm 2}, and Md Suruz Miah\textsuperscript{\rm 1}
\\  
\textsuperscript{\rm 1}Electrical and Computer Engineering, Bradley University, Peoria, IL, USA; emails:~\{blauer,nshepard\}@mail.bradley.edu,~smiah@bradley.edu\\
\textsuperscript{\rm 2}Computer Science, Math. and Science, St John's University, Queens, NY, USA; email:~keshtkaf@stjohns.edu
}
 \begin{document}

\maketitle

\begin{abstract}

  In this paper, we propose a design of a smart sanitizing wheeled mobile robot
  using ultra-violet (UV) sterilization technology. The current design of the
  robot includes exteroceptive sensors, such as sonars, a lidar, a radio-frequency
  transceiver, and a camera. A circular-shaped UV [wavelength is in the range of
  $200~\nano\meter$--$280~\nano\meter$ (C band)] LED light is mounted on top
  of the robot for the purpose of sterilization in sanitizing complex indoor
  environments. The proposed robot under development is expected to sanitize  
  complex environments smartly in the sense that the UV LED is mounted with an actuator so that the robot can
  easily adapt the height of the UV LED based on the map of the operating
  environment. The performance of the robot is
  initially tested using the commercial robot simulator in an indoor environment
  with various complexities.   

\end{abstract}

\section{Introduction}
\label{sec:introduction}

The design and development of a class of mobile robots in sterilizing indoor
environments, hospitals, offices, and other indoor spaces, for example, have
received a special attention in recent years, especially due to Coronavirus
diseases - 2019 (COVID-19)\todo[inline]{Reference}. Recently, a class of mobile robots has been
developed to sterilize indoor spaces using UV-C sterilization
technology~\todo[inline]{Reference}. Most of the mobile robots developed so far in
sanitizing indoor environments either rely on fixed-mount UV-tube or sprayers to
disinfect areas much like a conventional cleaning robot. It is worth noting
that, not many sanitizing robots use UV lights to disinfect robot based on the
map of the indoor operating environment. In this paper, we develop a prototype
of a modular and cost-effective mobile robot, where robot has the ability to
smartly decide on the indoor space to  be sterilize based on the map. The map of
the environment is created using sensor fusion technology much like what
conventional autonomous robots do using range and vision sensors.


\todo[inline]{Talk about some existing sanitizing robots available in the market}




\section{System Architecture}
\label{sec:SystemArchitecture}

\todo[inline]{Discuss two different architecture that we propose}

\begin{figure}[htpb]
  \centering
  \begin{subfigure}[b]{0.5\textwidth}
    \centering
    \includegraphics[scale=0.25]{../figs/img/frontViewScreenshotB}
    \caption{Overall design of the proposed smart sanitizing robot.}
    \label{fig:frontViewScreenshotB}
  \end{subfigure}
  \begin{subfigure}[b]{.5\textwidth}
    \centering
    \includegraphics[scale=0.1]{../figs/img/v2BodyDesign}
    \caption{Body design}
  \end{subfigure}
  \begin{subfigure}[b]{.5\textwidth}
    \centering
    \includegraphics[scale=0.1]{../figs/img/v2UVLamp}
    \subcaption{UV lamp model}
  \end{subfigure}
  \begin{subfigure}[b]{.5\textwidth}
    \centering
    \includegraphics[scale=0.1]{../figs/img/v2UVLampConveyor}
    \subcaption{UV lamp conveyor}
  \end{subfigure}
  \begin{subfigure}[b]{.5\textwidth}
    \centering
    \includegraphics[scale=0.1]{../figs/img/v2SprayerMount}
    \subcaption{Disinfectant sprayer mount}
  \end{subfigure}
  \caption{Design of Proposed Sanitizing Robots.}
  \label{fig:ArchitecturesOfSanitizingRobot}
\end{figure}


\subsection{Features}
\label{sec:Features}

\subsection{Operating Principle}
\label{sec:OperatingPrinciple}



\section{Detailed Design}
\label{sec:DetailedDesign}



\section{Real-Time Computer Experiment}
\label{sec:RealTimeComputerExperiment}


\section{Conclusion} \label{sec:conclusion}


% \bibliographystyle{aaai}
% \bibliography{bib/refsSuruzWeb,bib/refsMultiAgent,bib/refsRoboticsJournals,bib/refsRoboticsConferences,bib/refsGenericControl,bib/refsBooksTRTheses,bib/refsReinforcementLearningADP,bib/refsRL-Keshtkar}

\end{document}

%%% Local Variables:
%%% mode: latex
%%% TeX-master: t
%%% End:
